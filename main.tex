\documentclass[11pt,a4paper]{article}
% ^ review
%#########################################################################
\usepackage[utf8]{inputenc}
\usepackage[T1]{fontenc}
\usepackage[french]{babel}
%\usepackage[babel=true,kerning=true]{microtype}%évite le problème des ":"
\usepackage{lmodern}
\usepackage[hmargin=4cm, vmargin=2cm, includeheadfoot]{geometry}
\usepackage{alltt}
%\usepackage{csvsimple}
\usepackage{multicol}
\usepackage{amsmath,amssymb}
\usepackage{color}
\usepackage{graphicx}

\usepackage[francais,bloc]{automultiplechoice}
% ou bien
%\usepackage[correcindiv,francais,bloc]{automultiplechoice}

% -> [correcindiv] pour voir la solution
% -> [completemulti] pour ajouter "Aucune de ces réponses n'est correcte"

%\usepackage[francais,bloc,ensemble,outsidebox,automarks]{automultiplechoice}

%\usepackage{mhchem} % needed for chemical equations
%\usepackage{tikz} % inutile a priori
\usepackage{comment}

% exemple de commande utilisateur
\providecommand{\abs}[1]{\lvert#1\rvert}

% #########################################################################
% Entête
%#########################################################################

% \newcounter{cptexolabel}
% \stepcounter{cptexolabel}
% \newcommand{\exolabel}[1]{\vspace{1em}\par\noindent{\large\textbf{\underline{Exercice \thecptexolabel}} (#1)} \stepcounter{cptexolabel}\\}
% \newcommand{\datepreuve}{15/04/2019}
% \makeatletter
% \newcommand{\numcopie}{\the\AMCid@etud}
% \makeatother
% \def\multiSymbole{}
% \def\AMCbeginQuestion#1#2{\par\noindent{\bf Q#1:}#2}%\hspace*{1em}}
% \def\AMCformQuestion#1{\vspace{\AMCformVSpace}\par{\bf Q#1:}}

\newcommand{\ue}{WebX}

\title{QCM \ue}
\author{Equipe pédagogique de l'UE \ue{}}

%#########################################################################
% Document
%#########################################################################
\begin{document}

\shorthandoff{:}%

% BAREME 
% e=incohérence; b=bonne; m=mauvaise; p planché (on ne descent pas en dessous)
%% \baremeDefautM{formula=NBC/NB-NMC/NM,p=0}
\baremeDefautM{e=-1,b=1,m=-1,p=0}
\baremeDefautS{e=-1,b=1,m=0,p=0}
%\baremeDefautS{e=-0.5,b=1,m=-0.5}% never put b<1,
%\baremeDefautM{e=-0.5,b=1,m=-0.25,p=-0.5}% never put b<1, with amc2moodle m correspond to the grade if all the wrong answers are ticked, b correspond to the grade if all the good answers are ticked

%question : environnement, text question
%element{label}{groupe} :commande, encapsule la commande pour lui donner un groupe
%reponses : environnement
%bonne  : commande
%mauvaise : commande

% \begin{flushleft}
% \end{flushleft}
%\attachfile{./a.c}
%\includesounds{./Son/piano2.mp3} % test pour inclure sons
%\includegraphics[width=0.5\textwidth]{./Figures/other/schema_interpL.png}

%==============================================================================
\begin{comment} % TUTORIEL POUR L'EQUIPE PEDAGOGIQUE
%==============================================================================

% Objectif :
% - theme1 % 5 questions (par exemple)
% - theme2 % 5 questions
% - theme3 % 5 questions
% - theme4 % 5 questions

% QUESTION A CHOIX MULTIPLE :
\element{theme1}{
  \begin{questionmult}{Nom-question:QCM}
    SUJET...
    \begin{reponses}
      \bonne{REPONSE1}
      \bonne{REPONSE2}
      \mauvaise{AUTRES...}
    \end{reponses}
  \end{questionmult}
}

% QUESTION A CHOIX UNIQUE :
\element{theme1}{
  \begin{question}{Nom-question:QCU}
    SUJET...
    Cochez la bonne réponse (suggérer dans le sujet qu'il y a une
    unique réponse) :
    \begin{reponses}
      \bonne{REPONSE}
      \mauvaise{AUTRES...}
    \end{reponses}
  \end{question}
}

% Pour le verbatim, utiliser \texttt{ } ou bien le paquetage alltt.

%==============================================================================
\end{comment}
%==============================================================================

\element{theme1}{

  \begin{questionmult}{Intro:QCM}
    ENONCE1
    \begin{reponses}
      \bonne{OK1}
      \bonne{OK2}
      \mauvaise{KO}
    \end{reponses}
  \end{questionmult}

  \begin{questionmult}{Test:QCM}
    ENONCE2
    \begin{reponses}
      \bonne{OK}
      \mauvaise{KO1}
      \mauvaise{KO2}
    \end{reponses}
  \end{questionmult}
}

%%%%%%%%%%%%%%%%%%%%%%%%%%%%%%%%%%%%%%%%%%%%%%%%%%%%%%%%%%%%%%%%%%%%%%%%%%%%%%% 

% #################################################################
% C R E A T I O N  D E S  C O P I E S
% #################################################################
\exemplaire{1}{    	% nombre de sujet différent

  %debut de l'en-tête des copies :    

  \vspace*{.5cm}
  \begin{minipage}{.4\linewidth}
    \centering\large\bf Test
  \end{minipage}
  \champnom{\fbox{
      \begin{minipage}{.5\linewidth}
        Nom et prénom :

        \vspace*{.5cm}\dotfill
        \vspace*{1mm}
      \end{minipage}
    }}

  \begin{flushleft}
    \begin{center}
      \Large{\textsc{QCM \ue{}}}\\
      \normalsize
    \end{center}
  \end{flushleft}


  % groupe à exporter
  \cleargroup{2020}

  \copygroup{theme1}{2020}
  %\copygroup{theme2}{2020}
  % etc.

  %optional?% \melangegroupe{2020}
  \restituegroupe{2020}
}

\end{document}
